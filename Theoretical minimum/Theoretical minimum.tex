\documentclass[11pt]{article} 
\usepackage[utf8]{inputenc}
\usepackage[russian]{babel} 
\marginparwidth 0.5in 
\oddsidemargin 0.25in 
\evensidemargin 0.25in 
\marginparsep 0.25in 
\topmargin 0.25in 
\textwidth 6in \textheight 8 in 
\usepackage{amsmath} 
\usepackage{amssymb} 
\usepackage[T2A]{fontenc} 
\usepackage[utf8]{inputenc} 
\usepackage{graphicx} 
\usepackage{tikz} 
\usetikzlibrary{trees} 
\usepackage{verbatim} 
\DeclareGraphicsExtensions{.pdf,.png,.jpg} 
\usepackage{setspace}
\begin{document}


\section*{Вычислительная математика\\5 семестр\\}
\section*{Теоретический минимум }
\section*{Задание 1}
\hspace{17pt}1. Метод простой итерации (МПИ). Как записывается для одного уравнения и системы уравнений? Достаточное условие сходимости МПИ (для 1-го уравнения и системы).

2. Метод Ньютона: как записывается для одного уравнения и системы, условие сходимости метода Ньютона для одного уравнения. Порядок сходимости метода Ньютона, когда второй, в каких случаях первый, а в каких метод вообще расходится?

3. Метод деления отрезка пополам: вывести зависимость кол-ва итераций от точности, если задан отрезок локализации.

4. Численное диф-ие: задача на метод неопределенных коэффициентов или на порядок аппроксимации заданной формулы чилсенного диф-ия или на построение формулы производной путем диф-ия интерполяционного полинома.

5. Численное интегрирование: в чем принципиальное различие формул Ньютона-Котеса от квадратур Гаусса? Порядок точности для формул Ньютона-Котеса. Правило Рунге, экстраполяция Ричардсона. Как выводятся квадратурные формулы Гаусса?

\section*{Задание 2}

\hspace{17pt}1. Метод стрельбы. В чем его суть, как записыва.тся?

2. Метод прогонки для каких матриц? Достаточное условие сходимости метода прогонки.

3. Способы аппроксимации граничных условий: уметь решать задачу об аппроксимации условия на произвольную по двум приграничным точкам.

4. Метод Ньютона для решения краевых задач: знать суть метода, уметь его применять.

5. Методы Рунге-Куеты: как записывается, таблицы Бутчера, определение A-L-устойчивости

6. Формула функции устойчивости

7. Уметь линеаризовать

8. Определение жестких систем ОДУ

\end{document}

{\tt
\begin{tabbing}

\end{tabbing}
}